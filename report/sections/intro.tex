\section{Introduction}

Operators of social networks as well as companies are increasingly sharing information about their users. Would it be to support research or for commercial purposes, related data is typically protected by anonymization. Often, this "anonymization" is carried out by removing sensitive fields such as the name, address or Social Security Number of the user. Still, the scientific community has expressed doubt as to whether those methods guaranteed effective user privacy. Several successful attacks have been demonstrated, and this report aims to reproduce (with limitations such as reduced computing power capabilities) one of the most famous of those privacy breaches: the Netflix Prize dataset de-anonymization.

This report is structured as follows. First, in \autoref{sec:state}, a state of the art of de-anonymization techniques is presented. It summarizes the major existing attacks, then focuses on the Netflix case. Secondly, in \autoref{sec:approach}, our approach is detailed and our choices are described. Eventually, in \autoref{sec:results}, our results are shown, analyzed, and compared to the state of the art. 
